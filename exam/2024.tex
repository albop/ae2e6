% Options for packages loaded elsewhere
\PassOptionsToPackage{unicode}{hyperref}
\PassOptionsToPackage{hyphens}{url}
\PassOptionsToPackage{dvipsnames,svgnames,x11names}{xcolor}
%
\documentclass[
  letterpaper,
  DIV=11,
  numbers=noendperiod]{scrartcl}

\usepackage{amsmath,amssymb}
\usepackage{iftex}
\ifPDFTeX
  \usepackage[T1]{fontenc}
  \usepackage[utf8]{inputenc}
  \usepackage{textcomp} % provide euro and other symbols
\else % if luatex or xetex
  \usepackage{unicode-math}
  \defaultfontfeatures{Scale=MatchLowercase}
  \defaultfontfeatures[\rmfamily]{Ligatures=TeX,Scale=1}
\fi
\usepackage{lmodern}
\ifPDFTeX\else  
    % xetex/luatex font selection
\fi
% Use upquote if available, for straight quotes in verbatim environments
\IfFileExists{upquote.sty}{\usepackage{upquote}}{}
\IfFileExists{microtype.sty}{% use microtype if available
  \usepackage[]{microtype}
  \UseMicrotypeSet[protrusion]{basicmath} % disable protrusion for tt fonts
}{}
\makeatletter
\@ifundefined{KOMAClassName}{% if non-KOMA class
  \IfFileExists{parskip.sty}{%
    \usepackage{parskip}
  }{% else
    \setlength{\parindent}{0pt}
    \setlength{\parskip}{6pt plus 2pt minus 1pt}}
}{% if KOMA class
  \KOMAoptions{parskip=half}}
\makeatother
\usepackage{xcolor}
\setlength{\emergencystretch}{3em} % prevent overfull lines
\setcounter{secnumdepth}{-\maxdimen} % remove section numbering
% Make \paragraph and \subparagraph free-standing
\ifx\paragraph\undefined\else
  \let\oldparagraph\paragraph
  \renewcommand{\paragraph}[1]{\oldparagraph{#1}\mbox{}}
\fi
\ifx\subparagraph\undefined\else
  \let\oldsubparagraph\subparagraph
  \renewcommand{\subparagraph}[1]{\oldsubparagraph{#1}\mbox{}}
\fi

\usepackage{color}
\usepackage{fancyvrb}
\newcommand{\VerbBar}{|}
\newcommand{\VERB}{\Verb[commandchars=\\\{\}]}
\DefineVerbatimEnvironment{Highlighting}{Verbatim}{commandchars=\\\{\}}
% Add ',fontsize=\small' for more characters per line
\usepackage{framed}
\definecolor{shadecolor}{RGB}{241,243,245}
\newenvironment{Shaded}{\begin{snugshade}}{\end{snugshade}}
\newcommand{\AlertTok}[1]{\textcolor[rgb]{0.68,0.00,0.00}{#1}}
\newcommand{\AnnotationTok}[1]{\textcolor[rgb]{0.37,0.37,0.37}{#1}}
\newcommand{\AttributeTok}[1]{\textcolor[rgb]{0.40,0.45,0.13}{#1}}
\newcommand{\BaseNTok}[1]{\textcolor[rgb]{0.68,0.00,0.00}{#1}}
\newcommand{\BuiltInTok}[1]{\textcolor[rgb]{0.00,0.23,0.31}{#1}}
\newcommand{\CharTok}[1]{\textcolor[rgb]{0.13,0.47,0.30}{#1}}
\newcommand{\CommentTok}[1]{\textcolor[rgb]{0.37,0.37,0.37}{#1}}
\newcommand{\CommentVarTok}[1]{\textcolor[rgb]{0.37,0.37,0.37}{\textit{#1}}}
\newcommand{\ConstantTok}[1]{\textcolor[rgb]{0.56,0.35,0.01}{#1}}
\newcommand{\ControlFlowTok}[1]{\textcolor[rgb]{0.00,0.23,0.31}{#1}}
\newcommand{\DataTypeTok}[1]{\textcolor[rgb]{0.68,0.00,0.00}{#1}}
\newcommand{\DecValTok}[1]{\textcolor[rgb]{0.68,0.00,0.00}{#1}}
\newcommand{\DocumentationTok}[1]{\textcolor[rgb]{0.37,0.37,0.37}{\textit{#1}}}
\newcommand{\ErrorTok}[1]{\textcolor[rgb]{0.68,0.00,0.00}{#1}}
\newcommand{\ExtensionTok}[1]{\textcolor[rgb]{0.00,0.23,0.31}{#1}}
\newcommand{\FloatTok}[1]{\textcolor[rgb]{0.68,0.00,0.00}{#1}}
\newcommand{\FunctionTok}[1]{\textcolor[rgb]{0.28,0.35,0.67}{#1}}
\newcommand{\ImportTok}[1]{\textcolor[rgb]{0.00,0.46,0.62}{#1}}
\newcommand{\InformationTok}[1]{\textcolor[rgb]{0.37,0.37,0.37}{#1}}
\newcommand{\KeywordTok}[1]{\textcolor[rgb]{0.00,0.23,0.31}{#1}}
\newcommand{\NormalTok}[1]{\textcolor[rgb]{0.00,0.23,0.31}{#1}}
\newcommand{\OperatorTok}[1]{\textcolor[rgb]{0.37,0.37,0.37}{#1}}
\newcommand{\OtherTok}[1]{\textcolor[rgb]{0.00,0.23,0.31}{#1}}
\newcommand{\PreprocessorTok}[1]{\textcolor[rgb]{0.68,0.00,0.00}{#1}}
\newcommand{\RegionMarkerTok}[1]{\textcolor[rgb]{0.00,0.23,0.31}{#1}}
\newcommand{\SpecialCharTok}[1]{\textcolor[rgb]{0.37,0.37,0.37}{#1}}
\newcommand{\SpecialStringTok}[1]{\textcolor[rgb]{0.13,0.47,0.30}{#1}}
\newcommand{\StringTok}[1]{\textcolor[rgb]{0.13,0.47,0.30}{#1}}
\newcommand{\VariableTok}[1]{\textcolor[rgb]{0.07,0.07,0.07}{#1}}
\newcommand{\VerbatimStringTok}[1]{\textcolor[rgb]{0.13,0.47,0.30}{#1}}
\newcommand{\WarningTok}[1]{\textcolor[rgb]{0.37,0.37,0.37}{\textit{#1}}}

\providecommand{\tightlist}{%
  \setlength{\itemsep}{0pt}\setlength{\parskip}{0pt}}\usepackage{longtable,booktabs,array}
\usepackage{calc} % for calculating minipage widths
% Correct order of tables after \paragraph or \subparagraph
\usepackage{etoolbox}
\makeatletter
\patchcmd\longtable{\par}{\if@noskipsec\mbox{}\fi\par}{}{}
\makeatother
% Allow footnotes in longtable head/foot
\IfFileExists{footnotehyper.sty}{\usepackage{footnotehyper}}{\usepackage{footnote}}
\makesavenoteenv{longtable}
\usepackage{graphicx}
\makeatletter
\def\maxwidth{\ifdim\Gin@nat@width>\linewidth\linewidth\else\Gin@nat@width\fi}
\def\maxheight{\ifdim\Gin@nat@height>\textheight\textheight\else\Gin@nat@height\fi}
\makeatother
% Scale images if necessary, so that they will not overflow the page
% margins by default, and it is still possible to overwrite the defaults
% using explicit options in \includegraphics[width, height, ...]{}
\setkeys{Gin}{width=\maxwidth,height=\maxheight,keepaspectratio}
% Set default figure placement to htbp
\makeatletter
\def\fps@figure{htbp}
\makeatother

\KOMAoption{captions}{tableheading}
\makeatletter
\@ifpackageloaded{caption}{}{\usepackage{caption}}
\AtBeginDocument{%
\ifdefined\contentsname
  \renewcommand*\contentsname{Table of contents}
\else
  \newcommand\contentsname{Table of contents}
\fi
\ifdefined\listfigurename
  \renewcommand*\listfigurename{List of Figures}
\else
  \newcommand\listfigurename{List of Figures}
\fi
\ifdefined\listtablename
  \renewcommand*\listtablename{List of Tables}
\else
  \newcommand\listtablename{List of Tables}
\fi
\ifdefined\figurename
  \renewcommand*\figurename{Figure}
\else
  \newcommand\figurename{Figure}
\fi
\ifdefined\tablename
  \renewcommand*\tablename{Table}
\else
  \newcommand\tablename{Table}
\fi
}
\@ifpackageloaded{float}{}{\usepackage{float}}
\floatstyle{ruled}
\@ifundefined{c@chapter}{\newfloat{codelisting}{h}{lop}}{\newfloat{codelisting}{h}{lop}[chapter]}
\floatname{codelisting}{Listing}
\newcommand*\listoflistings{\listof{codelisting}{List of Listings}}
\makeatother
\makeatletter
\makeatother
\makeatletter
\@ifpackageloaded{caption}{}{\usepackage{caption}}
\@ifpackageloaded{subcaption}{}{\usepackage{subcaption}}
\makeatother
\ifLuaTeX
  \usepackage{selnolig}  % disable illegal ligatures
\fi
\usepackage{bookmark}

\IfFileExists{xurl.sty}{\usepackage{xurl}}{} % add URL line breaks if available
\urlstyle{same} % disable monospaced font for URLs
\hypersetup{
  pdftitle={Examen - Partie 2},
  colorlinks=true,
  linkcolor={blue},
  filecolor={Maroon},
  citecolor={Blue},
  urlcolor={Blue},
  pdfcreator={LaTeX via pandoc}}

\title{Examen - Partie 2}
\usepackage{etoolbox}
\makeatletter
\providecommand{\subtitle}[1]{% add subtitle to \maketitle
  \apptocmd{\@title}{\par {\large #1 \par}}{}{}
}
\makeatother
\subtitle{Macro II - Fluctuations - ENSAE, 2023-2024}
\author{}
\date{2024-04-10}

\begin{document}
\maketitle

\subsection{}\label{section}

On suppose que l'utilité du consommateur représentatif, l'individu \(i\)
est donnée par
\(\sum_{t=1}^T \frac{1}{(1+\rho)^t} \frac{(C_{it}/Z_{it})^(1-\theta)}{1-\theta}\)
avec \(\rho>0, \theta>0\) où \(Z_{it}\) est le niveau de référence de la
consommation. Son revenu est \(y_{i,t}\) et il peut épargner \(a_t\) à
un taux d'intérêt \(r\) supposant constant. On suppose qu'il n'y a pas
d'incertitude.

\textbf{Habitudes externes}: supposons \(Z_{it} = C_{t-1}^{\phi}\) avec
\(0\leq \phi \leq 1\). Cela signifie que le niveau de référence est
déterminé par la consommation agrégée, donc prise comme donnée par
l'individu.

\begin{enumerate}
\def\labelenumi{\arabic{enumi}.}
\item
  \textbf{Écrire la condition d'Euler pour la consommation. Exprimer
  \(\frac{C_{i,t+1}}{C_{i,t}}\) en fonction de \(\frac{C_t}{C_{t-1}}\)
  et \(\frac{1+r}{1+\rho}\).}
\item
  \textbf{A l'équilibre la consommation du consommateur représentatif
  vaut \(C_{i,t}=C_t\) pour tout \(t\). Utiliser ce fait pour écrire
  \(\log(C_{t+1}) - \log(C_t)\) en fonction de
  \(\log(C_t)-\log(C_{t-1})\). Pour \(\phi>0\) et \(\theta=1\), la
  formation d'habitude a-t-elle un effect sur le comportement de la
  consommation? Et pour \(\phi>0\) et \(\theta>1\)?}
\end{enumerate}

\textbf{Habitudes internes.} On suppose maintenant \(Z_{i,t}=C_{i,t}\).
C'est à dire que le niveau de consommation de référence de l'individu
est déterminée par sa consommation passé. On fixe \(\phi=1\).

\begin{enumerate}
\def\labelenumi{\arabic{enumi}.}
\setcounter{enumi}{2}
\item
  \textbf{Réécrire la condition d'Euler pour cette nouvelle
  spécification}.
\item
  \textbf{On note \(g_t=\frac{C_t}{C_{t-1}}-1\) la croissance de la
  consommation. Sous l'hypothèse, \(\rho=r=0\) et en supposant la
  croissance de la consommation proche de zero, donner une formule
  approchée au premier ordre liant \(g_{t+2}-g_{t+1}\) à
  \(g_{t+1}-g_t\)}.
\end{enumerate}

\subsection{}\label{section-1}

Le modfile de la page suivante correspond au modèle RBC en économie
ouverte.

\{\{newpage\}\}

\begin{Shaded}
\begin{Highlighting}[numbers=left,,]
\NormalTok{var y i c n a b k r w;}
\NormalTok{varexo epsilon;}
\NormalTok{parameters bet del alp nss khi eta rho rst;}

\NormalTok{bet=0.98;}
\NormalTok{alp=0.33;}
\NormalTok{del=0.025;}
\NormalTok{rho=0.95;}
\NormalTok{eta=1;}
\NormalTok{nss=0.33;}
\NormalTok{khi=(1{-}alp)*(1{-}nss)\^{}eta/nss*(1/bet{-}1+del)/(1/bet{-}1+del{-}del*alp);}
\NormalTok{rst=1/bet;}

\NormalTok{model;}
\NormalTok{1/c=bet*(r(1)+1{-}del)/c(1);}
\NormalTok{1/c=bet*rst/c(1);}
\NormalTok{w=khi*c/(1{-}n)\^{}eta;}
\NormalTok{k=(1{-}del)*k({-}1)+i;}
\NormalTok{y=a*k({-}1)\^{}alp*n\^{}(1{-}alp);}
\NormalTok{log(a)=rho*log(a({-}1))+epsilon;}
\NormalTok{w=(1{-}alp)*y/n;}
\NormalTok{r=alp*y/k({-}1);}
\NormalTok{b=y{-}c{-}i+rst*b({-}1);}
\NormalTok{end;}

\NormalTok{steady\_state\_model;}
\NormalTok{a=1;}
\NormalTok{r=1/bet{-}1+del;}
\NormalTok{n=nss;}
\NormalTok{k=(alp/r)\^{}(1/(1{-}alp))*n;}
\NormalTok{y=k\^{}alp*n\^{}(1{-}alp);}
\NormalTok{w=(1{-}alp)*y/n;}
\NormalTok{i=del*k;}
\NormalTok{c=y{-}i;}
\NormalTok{b=0;}
\NormalTok{end;}

\NormalTok{shocks;}
\NormalTok{var epsilon;stderr 0.009;}
\NormalTok{end;}

\NormalTok{check;}

\NormalTok{stoch\_simul(irf=200, order=1) y c i b;}
\end{Highlighting}
\end{Shaded}




\end{document}
